\documentclass[11pt]{article}
\usepackage{amsmath}
\usepackage{amssymb}
\usepackage{geometry}
\geometry{margin=1in}

\title{Basin Stability vs Thermal Equilibrium: \\
       Two Approaches to Phase Diagram Measurement}
\author{}
\date{}

\begin{document}
\maketitle

\section{Introduction}

When characterizing associative memory models, we must distinguish between two fundamentally different measurements:
\begin{enumerate}
\item \textbf{Thermal equilibrium statistics}: Measuring $\langle \varphi \rangle_{\text{eq}}$ in the Boltzmann distribution at temperature $T$
\item \textbf{Basin stability}: Testing whether perturbations around a pattern lead to successful retrieval
\end{enumerate}

These require different Monte Carlo protocols and probe different physics.

\section{Thermal Equilibrium Measurement}

\subsection{Protocol}
To measure thermal equilibrium alignment:
\begin{itemize}
\item Small initial perturbation: $\varepsilon \sim 0.05$
\item Long equilibration: $n_{\text{eq}} \sim 10^5$--$10^7$ steps (temperature-dependent)
\item Sampling after equilibration: $n_{\text{sample}} \sim 5000$ steps
\item Goal: System explores full Boltzmann distribution $p(\mathbf{x}) \propto e^{-E(\mathbf{x})/T}$
\end{itemize}

\subsection{Equilibration Time Requirement}
The system must explore phase space according to detailed balance. Equilibration time grows as:
\begin{equation}
\tau_{\text{eq}} \sim \exp\left(\frac{E_{\text{barrier}}}{T}\right)
\end{equation}

At low $T$, this becomes exponentially large, requiring adaptive protocols:
\begin{equation}
n_{\text{eq}}(T) = \max(10^5, \, C \cdot e^{-T/T_0})
\end{equation}

\subsection{Theoretical Prediction}
For thermal equilibrium in a confined basin (LSR case):
\begin{equation}
\langle \varphi \rangle_T = \frac{\int_{\varphi_c}^1 \varphi \, \rho(\varphi) \, e^{-E(\varphi)/T} \, d\varphi}{\int_{\varphi_c}^1 \rho(\varphi) \, e^{-E(\varphi)/T} \, d\varphi}
\end{equation}
where $\rho(\varphi) = (1-\varphi^2)^{(N-3)/2}$ is the density of states on $S^{N-1}(\sqrt{N})$, and $\varphi_c = (b-1)/b \approx 0.707$ is the hard wall in LSR.

\subsection{Observed Issue: Low-T Plateau}
Monte Carlo simulations with fixed equilibration time ($n_{\text{eq}} = 50{,}000$) show a plateau at low $T$:
\begin{itemize}
\item For $T < 0.19$: $\varphi \approx 0.9988$ (constant)
\item For $T > 0.2$: $\varphi(T)$ follows Boltzmann prediction (RMS error $\sim 2.4\%$)
\end{itemize}

\textbf{Interpretation}: At low $T$, acceptance rate drops dramatically. The system freezes in a quasi-equilibrated state before reaching true equilibrium. All low-$T$ runs converge to the same frozen state, creating the plateau.

\textbf{Conclusion}: Fixed equilibration time is insufficient for thermal equilibrium at low $T$.

\section{Basin Stability Measurement}

\subsection{Physical Question}
Instead of equilibrium thermodynamics, we ask: \textit{Is the retrieval state $\boldsymbol{\xi}^1$ stable against finite perturbations?}

This probes:
\begin{itemize}
\item Strength of the energy basin around $\boldsymbol{\xi}^1$
\item Competition between energy landscape (pulling toward $\boldsymbol{\xi}^1$) and thermal noise
\item Interference from spurious states at high $\alpha$
\end{itemize}

\subsection{Protocol}
\begin{itemize}
\item \textbf{Random initial alignment}: $\varphi_{\text{init}} \in [0.75, 1.0]$ (uniform distribution)
\item \textbf{Diverse initial states}: Each trial gets different random $\varphi_{\text{init}}$
\item \textbf{Relaxation dynamics}: $n_{\text{steps}} = 2^{14} = 16{,}384$ steps
\item \textbf{High statistics}: $n_{\text{trials}} = 1024$ trials
\item \textbf{Measurement}: Does $\mathbf{x}$ converge to $\boldsymbol{\xi}^1$ ($\varphi \to 1$) or diverge?
\end{itemize}

Key differences from equilibrium protocol:
\begin{enumerate}
\item \textbf{Controlled perturbation}: Initial states have $\varphi \in [0.75, 1.0]$, testing realistic perturbations
\item \textbf{Random diversity}: Each trial starts from different $\varphi$, not all identical
\item \textbf{Finite relaxation}: No need for infinite equilibration---just enough to see convergence direction
\item \textbf{Dynamical interpretation}: Measures basin of attraction, not thermal average
\end{enumerate}

\subsection{Implementation Details}

Initial states are constructed as follows:
\begin{enumerate}
\item For each trial $t$ and temperature $j$, choose random $\varphi_{\text{init}} \in [0.75, 1.0]$
\item Generate random unit vector $\mathbf{u}_\perp$ perpendicular to $\boldsymbol{\xi}^1$ (Gram-Schmidt)
\item Construct initial state:
\begin{equation}
\mathbf{x}_{\text{init}} = \varphi_{\text{init}} \cdot \boldsymbol{\xi}^1 + \sqrt{1 - \varphi_{\text{init}}^2} \cdot \sqrt{N} \cdot \mathbf{u}_\perp
\end{equation}
\end{enumerate}

This ensures:
\begin{itemize}
\item Exact control: $(\boldsymbol{\xi}^1 \cdot \mathbf{x}_{\text{init}})/N = \varphi_{\text{init}}$
\item Constraint satisfaction: $|\mathbf{x}_{\text{init}}| = \sqrt{N}$ (on the hypersphere)
\item For LSR: $\varphi_{\text{init}} \geq 0.75 > \varphi_c \approx 0.707$ (stays within basin)
\item Random diversity: Different trials test different perturbation magnitudes
\end{itemize}

\subsection{Physical Interpretation of $\langle \varphi \rangle$}

In the basin stability framework:
\begin{itemize}
\item $\varphi \approx 1$ (blue): Basin is stable, perturbations converge back to $\boldsymbol{\xi}^1$
\item $\varphi \approx 0$ (red): Basin is unstable or nonexistent
\item Intermediate $\varphi$: Partial success rate across trials
\end{itemize}

The phase boundary marks where basin stability is lost due to:
\begin{itemize}
\item \textbf{High $T$}: Thermal energy overcomes basin depth
\item \textbf{High $\alpha$}: Spurious attractors fragment phase space
\end{itemize}

\subsection{Why This Resolves the Plateau Issue}

With random initial states $\varphi_{\text{init}} \in [0.75, 1.0]$ and moderate relaxation:
\begin{itemize}
\item At low $T$: Strong energy gradient pulls system to $\varphi \to 1$ (retrieval)
\item At high $T$: Thermal noise prevents convergence, $\varphi$ varies with initial condition
\item \textbf{No freezing}: System starts with realistic perturbations, enough dynamics to reveal basin structure
\end{itemize}

The low-$T$ plateau disappears because:
\begin{enumerate}
\item System starts with significant perturbation ($\varphi_{\text{init}} \in [0.75, 1.0]$ instead of $0.999$)
\item Energy gradient is strong enough to drive convergence at low $T$
\item Random diversity tests basin robustness across different perturbation magnitudes
\item We're not trying to explore full Boltzmann distribution---just convergence direction
\end{enumerate}

\subsection{Physical Interpretation of Results}

The measured $\langle \varphi \rangle$ represents the **average final alignment** after relaxation:
\begin{itemize}
\item $\langle \varphi \rangle \approx 1$: Most trials converge to $\varphi \to 1$ regardless of $\varphi_{\text{init}}$ → \textbf{strong basin}
\item $\langle \varphi \rangle \in [0.75, 0.9]$: Mixed success; some trials converge, others drift → \textbf{marginal stability}
\item $\langle \varphi \rangle \approx 0.75$: Trials remain near initialization → \textbf{weak/no attraction}
\item For LSR at high $T$: $\langle \varphi \rangle \to 0.75$ (random walk within basin, no convergence)
\end{itemize}

\section{Connection Between Frameworks}

\subsection{When Do They Agree?}

At \textbf{moderate to high $T$ (above $T \sim 0.2$)}:
\begin{itemize}
\item Thermal fluctuations dominate initialization details
\item Basin stability $\approx$ thermal equilibrium in paramagnetic phase
\item Both frameworks give similar results
\end{itemize}

At \textbf{low $T$ (below $T \sim 0.2$)}:
\begin{itemize}
\item \textbf{Equilibrium framework}: Requires exponentially long equilibration (impractical)
\item \textbf{Basin stability}: Works naturally---strong gradient ensures convergence
\end{itemize}

\subsection{Physical Regimes}

\begin{center}
\begin{tabular}{l|l|l}
\hline
Regime & Basin Stability & Thermal Equilibrium \\
\hline
Low $T$, low $\alpha$ & Stable retrieval ($\varphi \to 1$) & $\langle \varphi \rangle_{\text{eq}} \to 1$ \\
High $T$, low $\alpha$ & Thermal destruction & $\langle \varphi \rangle_{\text{eq}} \to \varphi_c$ \\
Low $T$, high $\alpha$ & Spurious states interfere & Spin-glass phase \\
High $T$, high $\alpha$ & Failed retrieval & Paramagnetic/glassy \\
\hline
\end{tabular}
\end{center}

\section{Recommended Protocol: Basin Stability}

For characterizing the retrieval phase boundary, basin stability is more appropriate:

\begin{itemize}
\item \textbf{Initial alignment}: $\varphi_{\text{init}} \in [0.75, 1.0]$ (random, uniform distribution)
\item \textbf{Relaxation}: $n_{\text{steps}} = 2^{14} = 16{,}384$ steps
\item \textbf{Trials}: $n_{\text{trials}} = 1024$ (robust statistics)
\item \textbf{Temperature grid}: Log-spaced $10^{-2} \to 2.5$ (50 points) for both LSE and LSR
\item \textbf{$\alpha$ grid}: $0.01 : 0.01 : 0.55$ (55 values)
\item \textbf{Pattern count}: $P \in [500, 20{,}000]$ (linear in $\alpha$)
\end{itemize}

\subsection{Advantages}
\begin{enumerate}
\item \textbf{Physically relevant}: Directly tests retrieval capability
\item \textbf{Computationally efficient}: No exponential equilibration needed
\item \textbf{Works at all $T$}: No low-$T$ freezing issues
\item \textbf{Robust}: Averages over 1024 trials with diverse perturbations
\end{enumerate}

\section{Implementation: GPU-Accelerated Monte Carlo}

The basin stability protocol is implemented in:
\begin{itemize}
\item \texttt{basin\_stab\_LSE\_v1.jl}: LSE model (softmax energy)
\item \texttt{basin\_stab\_LSR\_v1.jl}: LSR model (ReLU energy with hard wall)
\end{itemize}

\subsection{Key Features}

\textbf{Parallel temperature simulation:}
\begin{itemize}
\item All $n_T = 50$ temperatures simulated simultaneously on GPU
\item Each temperature has $n_{\text{trials}} = 1024$ independent chains
\item Total: $51{,}200$ parallel chains per $\alpha$ value
\end{itemize}

\textbf{Initialization:}
\begin{verbatim}
# For each trial and temperature:
phi_init = rand(Uniform(0.75, 1.0))
x_perp = orthogonalize(randn(N), target)
x = phi_init * target + sqrt(1-phi_init^2) * sqrt(N) * x_perp
\end{verbatim}

\textbf{Relaxation:}
\begin{itemize}
\item Metropolis-Hastings on $S^{N-1}(\sqrt{N})$
\item Step size: $\sigma = \max(0.1, 2.4/\sqrt{N})$
\item LSR enforces hard wall: reject moves with $\varphi < \varphi_c \approx 0.707$
\end{itemize}

\textbf{Measurement:}
\begin{equation}
\langle \varphi \rangle_{\alpha, T} = \frac{1}{n_{\text{trials}}} \sum_{t=1}^{n_{\text{trials}}} \frac{1}{n_{\text{steps}}} \sum_{s=1}^{n_{\text{steps}}} \varphi_t^{(s)}
\end{equation}

where $\varphi_t^{(s)} = (\boldsymbol{\xi}^1 \cdot \mathbf{x}_t^{(s)})/N$ for trial $t$ at step $s$.

\subsection{Computational Performance}

Typical runtime on modern GPU (NVIDIA A100):
\begin{itemize}
\item Total grid: $55$ values of $\alpha \times 50$ values of $T = 2{,}750$ points
\item Each point: $1{,}024$ trials $\times$ $2^{14}$ steps
\item Total MC steps: $\sim 4.6 \times 10^{11}$
\item GPU time: $\sim 30$--$60$ minutes per model
\end{itemize}

\section{Conclusion}

The choice between equilibrium and basin stability protocols depends on the physical question:
\begin{itemize}
\item \textbf{Thermodynamics}: Use equilibrium protocol with adaptive equilibration
\item \textbf{Pattern retrieval}: Use basin stability protocol with large noise
\end{itemize}

For associative memory phase diagrams, \textbf{basin stability is the natural framework}. It directly measures what we care about: whether patterns can be successfully retrieved despite perturbations and thermal noise.

The excellent agreement between basin stability measurements and Boltzmann theory (at $T > 0.2$) confirms that both frameworks describe the same underlying physics---just from different perspectives.

\end{document}
