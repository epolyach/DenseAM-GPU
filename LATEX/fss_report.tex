\documentclass[11pt,a4paper]{article}
\usepackage[margin=2.5cm]{geometry}
\usepackage{amsmath,amssymb,amsthm}
\usepackage{graphicx}
\graphicspath{{../}}
\usepackage{booktabs}
\usepackage{caption}
\usepackage{subcaption}
\usepackage{hyperref}
\usepackage{xcolor}
\usepackage{float}

\title{Finite-Size Scaling for Dense Associative Memory\\Phase Boundaries on the $N$-Sphere:\\Metastability and the Limits of Linear Extrapolation}
\author{}
\date{}

\begin{document}
\maketitle

\section{Introduction}

We study the retrieval phase boundaries of Dense Associative Memory (DAM) models
on the $N$-sphere $S^{N-1}(\sqrt{N})$ using GPU-accelerated Metropolis--Hastings
Monte Carlo simulations. Two energy kernels are considered:
\begin{itemize}
    \item \textbf{LSE} (Gaussian / log-sum-exp): $E(\mathbf{x}) = -\frac{1}{\beta_{\mathrm{net}}}\log\sum_{\mu=1}^{P} e^{-\beta_{\mathrm{net}}(N - \boldsymbol{\xi}^\mu \cdot \mathbf{x})}$,
    \item \textbf{LSR} (Epanechnikov / log-sum-ReLU, $b=2+\sqrt{2}$):
    $E(\mathbf{x}) = -\frac{N}{b}\log\sum_{\mu=1}^{P}\max\!\Big(0,\; 1 - b + \frac{b}{N}\boldsymbol{\xi}^\mu\cdot\mathbf{x}\Big)$.
\end{itemize}
Here $\boldsymbol{\xi}^1,\dots,\boldsymbol{\xi}^P$ are random i.i.d.\ patterns
on $S^{N-1}(\sqrt{N})$, and the number of patterns scales exponentially:
$P = e^{\alpha N}$, where $\alpha = \ln P / N$ is the load parameter.

The goal is to map the order parameter $\varphi(\alpha, T)$ --- the overlap with
the target pattern --- across the $(\alpha, T)$ plane, and compare the observed
phase boundary with the theoretical prediction derived in the $N\to\infty$ limit.

\section{Monte Carlo Setup}

\subsection{Adaptive-$N$ approach}

For each value of $\alpha$, the number of neurons is determined by
\begin{equation}
    N(\alpha) = \left\lfloor \frac{\ln P(\alpha)}{\alpha} \right\rceil,
\end{equation}
where $P(\alpha)$ is interpolated linearly between $P_{\min}$ and $P_{\max}$
across the $\alpha$ grid ($\alpha \in [0.01, 0.55]$, step $0.01$; $T \in [0.05, 2.50]$, step $0.05$).

\subsection{Simulation parameters}

Each simulation uses:
\begin{itemize}
    \item $N_{\mathrm{eq}} = 5{,}000$ equilibration steps (no measurements),
    \item $N_{\mathrm{samp}} = 5{,}000$ sampling steps with overlap accumulation,
    \item Adaptive step size $\sigma = \max(0.1,\; 2.4/\sqrt{N})$,
    \item Initialization near the target pattern: $\mathbf{x}_0 = \boldsymbol{\xi}^1 + 0.05\,\boldsymbol{\eta}$, projected onto $S^{N-1}(\sqrt{N})$.
\end{itemize}
The overlap is averaged over multiple independent trials (each with freshly
generated random patterns).

\section{Finite-Size Scaling: Approach and Motivation}

\subsection{The problem at finite $N$}

The theoretical phase boundaries are derived in the thermodynamic limit
$N\to\infty$. At any finite $N$, the phase transition is broadened:
\begin{itemize}
    \item The overlap $\varphi$ does not jump discontinuously at the critical
    point but changes gradually over a region of width $\sim 1/\sqrt{N}$.
    \item The apparent critical temperature $T_c^{\mathrm{eff}}(N)$ differs
    from the theoretical $T_c(\infty)$.
\end{itemize}

Since $N = \ln P / \alpha$, at a given $\alpha$ one can increase $N$ only by
increasing $P$. Even increasing $P$ by orders of magnitude yields only modest
gains in $N$: for instance, at $\alpha = 0.10$, going from $P=1{,}000$ to
$P=300{,}000$ changes $N$ from 69 to 110 (a factor of $1.6\times$).

\subsection{Three P-scales}

We ran Monte Carlo sweeps at three progressively larger P-scales:

\begin{table}[H]
\centering
\begin{tabular}{lcccc}
\toprule
& $P$ range & Trials & $N$ at $\alpha=0.01$ & $N$ at $\alpha=0.55$ \\
\midrule
Scale~1 & 200 -- 5{,}000   & 256 & 530 & 15 \\
Scale~2 & 2{,}000 -- 50{,}000 & 256 & 760 & 20 \\
Scale~3 & 10{,}000 -- 300{,}000 & 64 & 921 & 23 \\
\bottomrule
\end{tabular}
\caption{Parameters for the three P-scales used in finite-size scaling.}
\label{tab:scales}
\end{table}

\subsection{Linear extrapolation to $N\to\infty$}

For each grid point $(\alpha, T)$, we have three values of $\varphi$ at three
different $N$ values. The finite-size scaling ansatz assumes
\begin{equation}\label{eq:fss}
    \varphi(N) = \varphi_\infty + \frac{c}{N} + O(1/N^2),
\end{equation}
where $\varphi_\infty$ is the thermodynamic-limit value. We perform a
least-squares linear fit of $\varphi$ vs.\ $1/N$ and extrapolate to $1/N = 0$.

\section{Results Across Scales}

\subsection{Scale~1: $P = 200$--$5{,}000$ (maps1.png)}

\begin{figure}[H]
\centering
\includegraphics[width=\textwidth]{maps1.png}
\caption{Scale~1 heatmaps. Left: LSE kernel. Right: LSR kernel ($b = 2+\sqrt{2}$).
Black curves show the theoretical phase boundaries.
The transition is broad, especially at large $\alpha$ where $N$ is small ($N\leq 31$).
Despite the broadening, the apparent boundary roughly follows the theoretical curve.}
\label{fig:maps1}
\end{figure}

At Scale~1, $N$ ranges from 530 (at $\alpha=0.01$) to 15 (at $\alpha=0.55$).
The transition is visibly broadened but the phase boundary location is
reasonable. At $\alpha = 0.10$, $N = 69$, and the transition occurs at
$T \approx 0.5$--$0.8$, close to the theoretical $T_c \approx 0.75$.

\subsection{Scale~2: $P = 2{,}000$--$50{,}000$ (maps2.png)}

\begin{figure}[H]
\centering
\includegraphics[width=\textwidth]{maps2.png}
\caption{Scale~2 heatmaps. The transition has sharpened somewhat compared to
Scale~1, but the apparent boundary has \emph{shifted upward} in $T$. The retrieval
region (red/warm colors) extends to higher temperatures than predicted by theory.}
\label{fig:maps2}
\end{figure}

At Scale~2, $N$ increases to 760 at $\alpha=0.01$ and 40 at $\alpha=0.25$.
The transition is sharper, but the apparent critical temperature $T_c^{\mathrm{eff}}$
is \emph{higher} than at Scale~1. For example, at $\alpha = 0.10$:
$\varphi(T\!=\!1.0) = 0.33$ (Scale~2) vs.\ $0.065$ (Scale~1), even though
$T\!=\!1.0$ is above the theoretical $T_c = 0.75$.

\subsection{Scale~3: $P = 10{,}000$--$300{,}000$ (maps3.png)}

\begin{figure}[H]
\centering
\includegraphics[width=\textwidth]{maps3.png}
\caption{Scale~3 heatmaps. The boundary has shifted even further from the
theoretical curve. The retrieval region extends well above $T_c$.}
\label{fig:maps3}
\end{figure}

Scale~3 has the largest $N$ values (921 at $\alpha=0.01$, 47 at $\alpha=0.25$),
yet the match with theory is \emph{worse}, not better. At $\alpha = 0.10$,
$T = 1.0$: $\varphi = 0.49$ (Scale~3), indicating the system appears to be
in the retrieval phase at a temperature well above $T_c = 0.75$.

\subsection{Extrapolated map (maps.png)}

\begin{figure}[H]
\centering
\includegraphics[width=\textwidth]{maps.png}
\caption{Extrapolated heatmaps ($N\to\infty$ via linear fit).
The extrapolation amplifies the metastability artifact, producing unphysical
$\varphi > 1$ values (clamped to 1.05) in the critical region.
234 LSE points and 473 LSR points exceed $\varphi = 1.0$.}
\label{fig:extrap}
\end{figure}

The linear extrapolation produces 234 unphysical points ($\varphi > 1$) for LSE
and 473 for LSR, concentrated in the critical region just above the theoretical
boundary.

\section{Diagnosis: Metastability and Critical Slowing Down}

\subsection{The first-order nature of the transition}

The retrieval--disordered phase transition in DAM models on the $N$-sphere is
\emph{first order}: the order parameter $\varphi$ jumps discontinuously at the
critical point in the thermodynamic limit. This means the free energy landscape
has two distinct minima (retrieval state with $\varphi > 0$ and disordered state
with $\varphi \approx 0$) separated by a \emph{free energy barrier} $\Delta F$.

For a first-order transition, this barrier grows with system size:
\begin{equation}
    \Delta F \sim N^{\gamma}, \quad \gamma > 0.
\end{equation}
The Kramers escape time from the metastable state scales as
\begin{equation}
    \tau_{\mathrm{escape}} \sim e^{\Delta F / T} \sim e^{N^\gamma / T},
\end{equation}
which grows \emph{exponentially} with $N$.

\subsection{Initialization bias}

Our Monte Carlo chains are initialized near the target pattern:
$\mathbf{x}_0 \approx \boldsymbol{\xi}^1$. This places the chain squarely in
the retrieval state. If the equilibrium state is actually the disordered phase
(above $T_c$), the chain must cross the free energy barrier to reach it.

\begin{itemize}
    \item At \textbf{small $N$} (Scale~1): The barrier is low. With $N_{\mathrm{eq}} = 5{,}000$
    steps, the chain can cross the barrier and equilibrate to the disordered state.
    Result: $\varphi \approx 0$ above $T_c$ (correct).

    \item At \textbf{larger $N$} (Scales~2,~3): The barrier is higher. With the
    same $N_{\mathrm{eq}} = 5{,}000$ steps, the chain \emph{cannot} escape the
    metastable retrieval state. Result: $\varphi$ remains elevated above $T_c$
    (incorrect --- metastability artifact).
\end{itemize}

\subsection{Quantitative evidence}

Table~\ref{tab:metastability} shows $\varphi$ across scales at points that
theory predicts to be in the disordered phase ($T > T_c$).

\begin{table}[H]
\centering
\begin{tabular}{llccccc}
\toprule
$\alpha$ & $T$ & $T_c$ (theory) & Scale~1 & Scale~2 & Scale~3 & Extrap. \\
\midrule
0.05 & 1.50 & 1.30 & 0.228 & 0.423 & 0.489 & \textcolor{red}{0.908} \\
0.10 & 0.80 & 0.75 & 0.601 & 0.672 & 0.678 & \textcolor{red}{0.824} \\
0.10 & 1.00 & 0.75 & 0.065 & 0.328 & 0.491 & \textcolor{red}{$>$1.05} \\
\midrule
\multicolumn{7}{l}{\textit{Correctly equilibrated points (far above $T_c$):}} \\
0.10 & 1.50 & 0.75 & $-$0.003 & 0.002 & 0.001 & 0.011 \\
0.10 & 2.50 & 0.75 & 0.000 & $-$0.001 & $-$0.001 & $-$0.002 \\
\bottomrule
\end{tabular}
\caption{Overlap $\varphi$ at points above the theoretical $T_c$. In the critical
region (top rows), $\varphi$ \emph{increases} with $N$, opposite to the expected
finite-size scaling. Far above $T_c$ (bottom rows), the chain equilibrates correctly.}
\label{tab:metastability}
\end{table}

The pattern is clear: $\varphi$ monotonically \emph{increases} with $N$ in the
critical region just above $T_c$, because the larger system is more deeply
trapped in the metastable retrieval state. The linear extrapolation
(\ref{eq:fss}) then projects this increasing trend to $1/N = 0$, producing
the unphysical $\varphi > 1$.

\subsection{Why the boundary shifts upward}

The visual effect on the heatmap is that the apparent phase boundary shifts to
higher $T$ with increasing $N$. This is the opposite of what standard finite-size
scaling predicts (boundary converging toward theory). The mechanism is:
\begin{enumerate}
    \item In the \textbf{retrieval phase} (below $T_c$): $\varphi$ is high and
    nearly $N$-independent. Correct.
    \item In the \textbf{critical region} ($T \lesssim T_c + \Delta T$): the MC
    chain remains trapped in the retrieval state at larger $N$.
    The apparent $\varphi$ stays high $\Rightarrow$ boundary appears at higher $T$.
    \item \textbf{Deep in the disordered phase} ($T \gg T_c$): thermal fluctuations
    are strong enough to kick the chain out of the metastable state even at large $N$.
    $\varphi \approx 0$ for all scales. Correct.
\end{enumerate}

The width of the ``metastability band'' above $T_c$ where the MC fails to
equilibrate \emph{grows} with $N$, explaining the worsening match with theory.

\section{Why Linear Extrapolation Fails}

The finite-size scaling ansatz $\varphi(N) = \varphi_\infty + c/N$ assumes that
the MC correctly samples the equilibrium Gibbs distribution at each $N$. When
metastability violates this assumption, the measured $\varphi(N)$ is
\emph{not the equilibrium value} but a metastable artifact:
\begin{equation}
    \varphi_{\mathrm{measured}}(N) = \varphi_{\mathrm{eq}}(N) + \Delta\varphi_{\mathrm{meta}}(N),
\end{equation}
where $\Delta\varphi_{\mathrm{meta}}(N)$ increases with $N$ in the critical
region. The linear fit then extrapolates the sum, and since
$\Delta\varphi_{\mathrm{meta}}$ grows with $N$ (decreases with $1/N$), the
extrapolated $\varphi_\infty$ overshoots.

The severity depends on the region:
\begin{itemize}
    \item \textbf{Deep retrieval:} $\Delta\varphi_{\mathrm{meta}} \approx 0$
    (system is in correct state). Extrapolation valid.
    \item \textbf{Critical region:} $\Delta\varphi_{\mathrm{meta}}$ dominates.
    Extrapolation fails catastrophically ($\varphi > 1$).
    \item \textbf{Deep disordered:} $\Delta\varphi_{\mathrm{meta}} \approx 0$
    (thermal fluctuations overcome barrier). Extrapolation valid.
\end{itemize}

\section{Possible Remedies}

\subsection{Parallel tempering (replica exchange)}
Run multiple replicas at different temperatures simultaneously. Allow replicas to
exchange configurations, enabling the system to escape metastable states by
``borrowing'' thermal energy from higher-temperature replicas. This is the most
robust approach for first-order transitions.

\subsection{Dual-initialization protocol}
For each $(\alpha, T)$ point, run the MC from two starting conditions:
(a)~near the target pattern (testing retrieval stability) and
(b)~from a random point on $S^{N-1}$ (testing whether the system spontaneously
finds the retrieval state). If the two give different $\varphi$, the system is
in the metastable/hysteresis region and $N_{\mathrm{eq}}$ is insufficient.

\subsection{Binder cumulant analysis}
Instead of extrapolating $\varphi$ directly, compute the Binder cumulant
$U_4 = 1 - \langle \varphi^4\rangle / (3\langle\varphi^2\rangle^2)$
at each scale. The crossing point of $U_4(T)$ curves for different $N$ gives
$T_c$ without requiring equilibration in the correct phase.

\subsection{Adaptive equilibration}
Scale $N_{\mathrm{eq}}$ with $N$, e.g., $N_{\mathrm{eq}} \propto N^2$ or even
exponentially. This is computationally expensive but addresses the root cause.

\subsection{Simulated annealing initialization}
Instead of starting near the target, start from a random configuration and
slowly cool the system (decrease $T$) in a pre-equilibration phase. This avoids
the initialization bias entirely, but may fail to find the retrieval state at low
$T$ if the basin of attraction is narrow.

\section{Conclusions}

The finite-size scaling approach---running MC simulations at three P-scales and
linearly extrapolating $\varphi$ vs.\ $1/N$---was designed to remove finite-$N$
artifacts and converge to the theoretical $N\to\infty$ phase boundary. However,
the approach fails in the critical region due to \textbf{metastability}:

\begin{enumerate}
    \item The retrieval--disordered transition is first order, with a free energy
    barrier that grows with $N$.
    \item MC chains initialized near the target pattern become exponentially
    harder to equilibrate at larger $N$ when $T > T_c$.
    \item The measured $\varphi$ in the critical region increases with $N$
    (opposite to equilibrium finite-size scaling), and the linear extrapolation
    amplifies this artifact.
\end{enumerate}

The result is that larger P-scales produce \emph{worse} agreement with theory
near the phase boundary, and the extrapolated map contains unphysical values
($\varphi > 1$). Addressing this requires fundamentally different MC strategies
(parallel tempering, dual initialization) rather than simply increasing system
size.

\end{document}
