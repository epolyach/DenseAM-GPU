\documentclass[11pt]{article}
\usepackage{amsmath}
\usepackage{amssymb}
\usepackage{geometry}
\usepackage{bm}
\usepackage{graphicx}
\usepackage{hyperref}
\geometry{margin=1in}

\newcommand{\vp}{\varphi}
\newcommand{\bS}{\mathbf{S}}
\newcommand{\bxi}{\boldsymbol{\xi}}

\title{Basin Percolation in the LSR Associative Memory}
\author{}
\date{}

\begin{document}
\maketitle

\section{Setup: LSR energy and the hard wall}

The Log-Sum-ReLU (LSR) associative memory with $M$ patterns
$\bxi^1, \dots, \bxi^M$ on the sphere $S^{N-1}(\sqrt{N})$
has energy
\begin{equation}\label{eq:E_LSR}
  E(\bS) = -\frac{N}{b}\,\ln \sum_{\mu=1}^{M}
    \bigl[\,1 - b + b\,\vp_\mu(\bS)\,\bigr]_+,
  \qquad b = 2 + \sqrt{2},
\end{equation}
where $\vp_\mu(\bS) = (\bxi^\mu \cdot \bS)/N$ is the overlap
and $[\cdot]_+ = \max(0, \cdot)$.

The hard wall activates at $\vp_c = (b-1)/b = 1/\sqrt{2} \approx 0.707$:
pattern $\mu$ contributes to the energy only when
$\vp_\mu(\bS) > \vp_c$.  When no pattern is active, the argument
of the logarithm vanishes and $E = +\infty$.

\paragraph{Basin of a single pattern.}
For an isolated target $\bxi^1$, the accessible region is the
spherical cap
\begin{equation}\label{eq:basin}
  X_0 = \bigl\{\, \bS \in S^{N-1}(\sqrt{N})
    \;:\; \vp_1(\bS) > \vp_c \,\bigr\},
\end{equation}
with an infinite energy barrier at the boundary $\vp_1 = \vp_c$.
The state is permanently confined to $X_0$.


\section{Joint basins and continuous connectivity}

With $M$ patterns, the accessible region is the set where
$\sum_\mu [\cdots]_+ > 0$, i.e.\ where at least one pattern
is active.  Two individual basins $X_\mu$ and $X_\nu$ form a
\emph{joint basin} if their intersection
$X_\mu \cap X_\nu \neq \varnothing$.

Along any continuous path through $X_\mu \cap X_\nu$,
both terms are simultaneously positive, so the energy
is always finite---the state never encounters $\ln 0$.
The energy does increase at the ``bottleneck'' (narrowest
part of the overlap corridor), creating a finite barrier
\begin{equation}\label{eq:barrier}
  \Delta E \;\sim\; \frac{N}{2b}\,\ln\frac{1}{q},
\end{equation}
where $q = (\bxi^\mu \cdot \bxi^\nu)/N$ is the mutual overlap,
but this barrier is finite for any $q > 0$.

\paragraph{Geometric overlap criterion.}
Two spherical caps of angular radius $\theta_c = \arccos\vp_c = \pi/4$
overlap when the angle between their centers satisfies
$\theta < 2\theta_c = \pi/2$, i.e.\ when
\begin{equation}\label{eq:geom_overlap}
  q = \frac{\bxi^\mu \cdot \bxi^\nu}{N} > \cos\!\bigl(\tfrac{\pi}{2}\bigr) = 0.
\end{equation}
Since random patterns have $q \sim \mathcal{N}(0, 1/N)$,
approximately $M/2$ patterns satisfy $q > 0$ geometrically.
However, the energy barrier~\eqref{eq:barrier} scales as
$\sim (N/4b)\ln N$ for $q \sim 1/\sqrt{N}$, making such
corridors thermally inaccessible.


\section{Percolation framework}

\subsection{Thermal connectivity threshold}

We define two basins as \emph{thermally connected} if the
energy barrier along the optimal path through their overlap
is comparable to the thermal energy scale $T$.  This requires
a minimum mutual overlap $q > q_\mathrm{eff}(T)$.

At zero temperature, only basins with a wide, low-barrier
corridor are connected.  The natural hard-wall threshold
is $q_\mathrm{eff} = \vp_c$, since patterns with
$q > \vp_c$ are already active at the target center
(the term $[1 - b + b\,q]_+> 0$).

\subsection{Poisson model for the number of neighbors}

For random patterns on $S^{N-1}(\sqrt{N})$, the overlap
$q = (\bxi^\mu \cdot \bxi^1)/N$ with a fixed target is
approximately $\mathcal{N}(0, 1/N)$ for large $N$.
The number of patterns with $q > \vp_c$ follows a Poisson
distribution with rate
\begin{equation}\label{eq:lambda}
  \lambda = M \cdot \Pr(q > \vp_c)
  \;\approx\;
  \frac{\exp\!\bigl(N(\alpha - \vp_c^2/2)\bigr)}
       {\vp_c \sqrt{2\pi N}},
\end{equation}
where $M = e^{N\alpha}$ and we used the Mill's ratio
approximation for the Gaussian tail.

\paragraph{Finite-$N$ correction.}
The exact overlap distribution on $S^{N-1}$ has density
$f(q) \propto (1 - q^2)^{(N-3)/2}$, which has a lighter tail
than the Gaussian for moderate $N$.  The Mill's ratio
therefore \emph{overestimates} $\lambda$; the true percolation
threshold is shifted to higher $\alpha$ for finite $N$.
This correction vanishes as $N \to \infty$.

\subsection{Percolation transition}

The basins of $M$ random patterns form a random geometric graph
on $S^{N-1}$: each basin is a node, and two nodes are connected
if their mutual overlap exceeds $q_\mathrm{eff}$.
The connected component containing the target undergoes a
\textbf{percolation transition} at $\lambda = 1$.

Setting $\lambda = 1$ in \eqref{eq:lambda} with
$q_\mathrm{eff} = \vp_c$:
\begin{equation}\label{eq:alpha_c}
  \boxed{\alpha_c = \frac{\vp_c^2}{2} + \frac{\ln(\vp_c\sqrt{2\pi N})}{N}
    \;\;\xrightarrow{N\to\infty}\;\;
    \frac{\vp_c^2}{2} = \frac{1}{4} = 0.25\,.}
\end{equation}
This coincides exactly with the zero-temperature critical capacity
$\alpha_\mathrm{th} = \tfrac{1}{2}(1 - 1/b)^2 = \tfrac{1}{2}\vp_c^2$
from the mean-field theory.

\paragraph{Physical picture.}
\begin{itemize}
  \item \textbf{Subcritical} ($\alpha < \alpha_c$):
    $\lambda < 1$; the target's connected cluster has $O(1)$
    patterns.  The state is confined near the target $\to$ retrieval.
  \item \textbf{Critical} ($\alpha \approx \alpha_c$):
    $\lambda \approx 1$; cluster size diverges; critical fluctuations.
  \item \textbf{Supercritical} ($\alpha > \alpha_c$):
    $\lambda > 1$; a giant connected component forms.
    The state delocalizes among exponentially many basins
    $\to$ paramagnetic phase.
\end{itemize}

\subsection{Temperature dependence}

At temperature $T > 0$, thermal fluctuations lower the effective
barrier, so narrower corridors become traversable:
$q_\mathrm{eff}(T) < \vp_c$.  The percolation threshold
shifts to
\begin{equation}
  \alpha_c(T) \approx \frac{q_\mathrm{eff}(T)^2}{2},
\end{equation}
which decreases with $T$---the retrieval region shrinks,
consistent with the phase diagram.

From the mean-field free energy calculation:
$\alpha_c(T) = \tfrac{1}{2}[1 - f_\mathrm{ret}(T)]^2$,
where $f_\mathrm{ret}(T) = u(\vp(T)) - T\,s(\vp(T))$ is
the retrieval free energy.  Identifying
$q_\mathrm{eff}(T) = 1 - f_\mathrm{ret}(T)$ connects
the percolation threshold to the thermodynamic phase boundary.


\section{BFS depth: chain connections}

Starting from the target, we explore the basin connectivity
graph via breadth-first search:
\begin{enumerate}
  \item \textbf{Depth~0}: target pattern $\bxi^1$ with basin $X_0$.
  \item \textbf{Depth~1}: all patterns $\mu$ with
    $(\bxi^\mu \cdot \bxi^1)/N > \vp_c$
    ($K_1 \sim \mathrm{Poisson}(\lambda)$ patterns).
  \item \textbf{Depth~2}: for each depth-1 pattern $\mu$,
    find patterns $\nu$ with $(\bxi^\mu \cdot \bxi^\nu)/N > \vp_c$
    that are \emph{not} already at depth~1.
\end{enumerate}

\paragraph{Depth-2 patterns are absent for large $N$.}
For a depth-1 pattern $\mu$ (overlap $\vp_\mu > \vp_c$ with target)
and a candidate depth-2 pattern $\nu$ (overlap $\vp_\nu < \vp_c$
with target), their mutual overlap is
\begin{equation}
  \frac{\bxi^\mu \cdot \bxi^\nu}{N}
  \;\approx\; \vp_\mu \cdot \vp_\nu
  + \sqrt{(1-\vp_\mu^2)(1-\vp_\nu^2)}\;\frac{u_\mu \cdot u_\nu}{\sqrt{N-1}},
\end{equation}
where $u_\mu, u_\nu$ are random unit vectors in the $(N{-}1)$-dimensional
subspace perpendicular to the target.

The deterministic part satisfies
$\vp_\mu \cdot \vp_\nu < \vp_c^2 = 1/2 < \vp_c = 1/\sqrt{2}$,
and the random fluctuation $\sim \mathcal{O}(1/\sqrt{N})$ cannot bridge the
gap $\vp_c - \vp_c^2 \approx 0.207$ for large $N$.

At finite $N$, the fluctuation has standard deviation
$\sim 0.5/\sqrt{N-1}$, requiring a $\sim 0.207\sqrt{N}$~sigma
event for a depth-2 connection.  For $N = 25$: $\sim 1\,\sigma$
(rare but possible); for $N = 50$: $\sim 1.5\,\sigma$ (very rare);
for $N \geq 100$: negligible.


\section{Simulation design}

The accompanying script \texttt{percolation\_LSR.jl}
tests four aspects of the percolation hypothesis:

\begin{enumerate}
  \item \textbf{Panel~1: Analytical $\lambda(\alpha)$}
    for $N = 30, 50, 75, 150$.
    All curves cross $\lambda = 1$ near $\alpha \approx 0.25$,
    with finite-$N$ corrections $\sim \ln N / N$ shifting
    the threshold to higher $\alpha$.

  \item \textbf{Panel~2: Direct neighbor counting}
    ($N = 25$, 3000 realizations).
    Generate $M = e^{N\alpha}$ random patterns on $S^{N-1}(\sqrt{N})$,
    count neighbors $K$ with overlap $> \vp_c$.
    Compare histogram with $\mathrm{Poisson}(\lambda)$.
    The Mill's ratio approximation overestimates $\lambda$ at
    moderate $N$ because the exact spherical tail is lighter
    than the Gaussian.

  \item \textbf{Panel~3: Branching process survival}
    ($N = 50$).
    Simulate a Galton--Watson process with
    $\mathrm{Poisson}(\lambda)$ offspring.
    The survival probability transitions from 0 to 1 at
    $\alpha_c$, sharpening with BFS depth (analogous to
    increasing system size in percolation).

  \item \textbf{Panel~4: BFS depth analysis}
    ($N = 25$, 200 realizations).
    Count depth-1 neighbors $K_1$ and \emph{new} depth-2
    neighbors $K_2$.
    Confirms that $\langle K_2 \rangle \ll \langle K_1 \rangle$:
    the BFS effectively terminates at depth~1.
\end{enumerate}


\section{Implications for the simulation protocol}

The percolation picture justifies the Poisson pattern reduction
scheme (v5 approach) with an important correction:
\textbf{the background energy $S_\mathrm{bg}$ should be removed}.

\begin{enumerate}
  \item The energy computed from just the $K+1$ retained patterns
    (target $+$ $K$ active neighbors) is \emph{exact} for all states
    reachable from the target basin.  Patterns outside the connected
    component are separated by infinite barriers and cannot be reached.

  \item The $S_\mathrm{bg}$ term in v5 artificially smooths the energy
    landscape, removing the infinite barriers that confine the state.
    This distorts the dynamics near the phase boundary.

  \item With the Poisson scheme (no $S_\mathrm{bg}$), memory scales as
    $\mathcal{O}(N \cdot K_\mathrm{max})$ instead of
    $\mathcal{O}(N \cdot M)$, enabling much larger $N$
    for sharper phase boundaries.

  \item For $\alpha > \alpha_c$ (supercritical), $K$ grows
    exponentially but can be capped at some $K_\mathrm{cap}$
    without affecting the measured overlap $\vp$ at the target---the
    state delocalizes among the $K_\mathrm{cap}$ nearest basins,
    which is sufficient to show non-retrieval.
\end{enumerate}

\paragraph{Connection to the Hopfield model.}
In the classical Hopfield model, the coupling structure creates an
exponentially large number of spin-glass metastable states via
random-matrix frustration.  In the LSR model, the $\log$-$\sum$-$\exp$
energy function confines the state to discrete basins separated by
infinite barriers.  The phase transition is purely
\emph{geometric}---basin percolation---rather than arising
from frustrated interactions.  This explains the absence of the
spin-glass phase in dense associative memories.


\begin{thebibliography}{9}
\bibitem{Amit1987}
  D.~J.~Amit, H.~Gutfreund, and H.~Sompolinsky,
  ``Statistical mechanics of neural networks near saturation,''
  \textit{Ann.\ Phys.} \textbf{173}, 30--67 (1987).

\bibitem{Penney1993}
  R.~W.~Penney, A.~C.~C.~Coolen, and D.~Sherrington,
  ``Coupled dynamics of fast spins and slow interactions
  in neural networks and spin systems,''
  \textit{J.\ Phys.\ A} \textbf{26}, 3681 (1993).

\bibitem{MezardParisi1986}
  M.~M{\'e}zard, G.~Parisi, and M.~A.~Virasoro,
  \textit{Spin Glass Theory and Beyond}, World Scientific, 1986.
\end{thebibliography}

\end{document}
